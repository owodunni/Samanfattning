\documentclass[a4paper,12pt]{article}

\title{Mekanik I, kapitel 1, Fö 1-2}
\author{A.O Poole}
%% Definitioner för vågfysikrapporten-dokument

%% Text-kodning, språk samt PS-font
\usepackage[utf8]{inputenc}
\usepackage[T1]{fontenc}
\usepackage{ae,aecompl}
\usepackage{listings}
% % bitmap-grafik
\usepackage{graphicx}
% % matematik
\usepackage{mathtools}
\usepackage{latexsym}
\usepackage{graphicx}

%% Paragrafformat
\setlength{\parindent}{0pt}
\setlength{\parskip}{1ex plus 0.5ex minus 0.2ex}

%% Format för datum
\newcommand{\twodigit}[1]{\ifthenelse{#1<10}{0}{}{#1}}
\newcommand{\dagensdatum}{
\number\year-\twodigit{\number\month}-\twodigit{\number\day}}

%% Sidhuvud och sidfot
\let\runtitle@title
\let\runauthor@author
\makeatother
\usepackage{fancyhdr}
\pagestyle{fancy}
\lhead{\runtitle}
\rhead{\runauthor}
\lfoot{alepo020@student.liu.se}
\cfoot{{\ } \\ \thepage}


%%Gray box
\usepackage{mdframed}




%%Dokumentets början
\begin{document}

\section{Punkters Kinematik}

   %\[\abs{\Value}  \quad \norm{\Value}  \qquad\text{non-starred}  \]
    %\[\abs*{\Value} \quad \norm*{\Value} \qquad\text{starred}\qquad\]

OBS! Ska man vara korrekt då man pratar om initsialramar så ska man betäckna vilken
referensram lägesvektorn $r$ tillhör d.v.s. $r_r$. Detta struntar jag dock i då det
inte behövs på tentan. Vill man göra det ändå så set det ut på följande vis 
$r = r_r$, $v = v_r$ och $a = a_r$.

\begin{itemize}
  \item Referensram - stel odeformerbar kropp, man kollar ofta hur något rör sig i 
  förhållande till en referensram.I en given referensram $r$ kan man införa ett 
  kordinatsystem med basvektorerna $\hat{e}^r_i$ kan vi beskriva lägesvektorn 
  $r_r$ för en punkt i rörelse.

\end{itemize}

\begin{equation}
 r = \sum\limits_{i=1}^n r\hat{e}^r_i 
\end{equation}

\begin{equation}
v = \frac{dr}{dt} = \sum\limits_{i=1}^n \dot{r}\hat{e}^r_i 
\end{equation}

\begin{equation}
a = \frac{d^2r}{dt^2} = \sum\limits_{i=1}^n \ddot{r}\hat{e}^r_i 
\end{equation}


\subsection{Plana kurvors geometri}

\begin{itemize}
  \item En kurva kan alltid parametriseras med avsende på en skalär dvs $u(t)$
        är entydligt bestämt.
\end{itemize}

\[ \hat{t} = \frac{dr}{ds}$ \]

\hat{n} = \frac{\frac{d\hat{t}}{ds}}{\abs{\frac{d\hat{t}}{ds}}}}


\begin{equation}
a = \frac{d^2r}{dt^2} = \summ\limits_{i=1}^n \ddot{r}\hat{e}^r_i 
\end{equation}

\subsection{ordlista}

\begin{itemize}

  \item Skalär - reala tal

  \item 

  \item 

  \item

\end{itemize}

\end{document}


